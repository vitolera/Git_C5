\documentclass[9pt]{beamer}
\usetheme{Rochester}
\usepackage[utf8]{inputenc}
\usepackage[spanish]{babel}
\usepackage{amsmath}
\usepackage{amsfonts}
\usepackage{amssymb}
\usepackage{graphicx}
\author{renatolrr}
\title{Bibliografía en Latex}
%\setbeamercovered{transparent} 
%\setbeamertemplate{navigation symbols}{} 
%\logo{} 
%\institute{} 
\date{} 
%\subject{} 
\begin{document}

\begin{frame}
\titlepage
\end{frame}

\begin{frame}
\tableofcontents
\end{frame}
\section{Creación de bases de datos para BibTeX}
\begin{frame}{Creación de bases de datos para BibTeX}

La forma mas cómoda de trabajar con bibliografías, es crearnos nuestra base de datos, con la ventaja de que así evitamos tener que crear una para cada vez que la necesitemos.

Con este método las referencias son escritas una sola vez y almacenadas en la base de datos que tenemos.

Pero como toda base de datos que se precie, tenemos que usar un formato que esta ya especificado, vamos a conocerlo.
\end{frame}

\begin{frame}{Creación de bases de datos para BibTeX}
@BOOK{osl:03a,\\ 	
AUTHOR 	="Oficina de Software Libre",\\
TITLE 	={Liberación de Software},\\
EDITION 	="primera",\\
PUBLISHER 	="Editorial UGR",\\
ADDRESS 	={Granada, GR},\\
YEAR 	=2010 }\\
\end{frame}

\begin{frame}{Creación de bases de datos para BibTeX}
Tipos de referencias\\
Campos requeridos\\
Campos opcionales\\

@article Artículos en revistas\\
	autor, title, journal, year.\\
	volume, number, pages, month, note\\
@book Libros con editorial conocida\\
 	author or editor, title, publisher, year\\
	volume or number, series, address, edition, month, note\\
@booklet Libros sin conocimiento de la editorial que lo publique\\
	title\\
	author, howpublished, address, month, year, note\\
@conference Artículo en un recopilatorio de una conferencia\\
	author, title, booktitle, year\\
	editor, volume or number, series, pages, address, month, organisation, publisher, note\\
\end{frame}

\begin{frame}{Creación de bases de datos para BibTeX}	
@inbook Entrada para una parte de un libro\\
	author or editor, title, chapter and/or pages, publisher, year\\
	volume or number, series, type, address, edition, month, note\\
@incollection Entrada para una parte de un libro con título propio\\
	author, title, booktitle, publisher, year\\
	editor, volume or number, series, type, chapter, pages, address, edition, month, note\\
@inproceedings Artículo en las publicaciones de un congreso\\
	author, title, booktitle, year\\
	editor, volume or number, series, pages, address, month, organisation, publisher, note\\
@manual Entrada para documentación de tipo técnico\\
	title\\
	author, organisation, address, edition, month, year, note\\
@masterthesis Entrada para proyecto, tesina o master\\
	author, title, school, year\\
	type, address, month, note\\
	\end{frame}
	
\begin{frame}{Creación de bases de datos para BibTeX}
@misc Documento que no se ajusta a ninguno de los demás tipos\\
	none\\
	author, title, howpublished, month, year, note\\
@phdthesis Tesis doctoral\\
	author, title, school, year\\
	type, address, month, note\\
@proceedings Recopilatorio de artículos de una conferencia o congreso\\
	title, year\\
	editor, volume or number, series, address, month, organisation, publisher, note\\
@unpublished Documento no publicado con titulo y autor\\
	author, title, note\\
	month, year\\
	
\end{frame}
\section{Gestión de base de datos}
\begin{frame}{Gestión de base de datos}
La creación y mantenimiento de una base de datos .bib puede resultar muy simple, si usamos algunos programas diseñados para manejar este tipo de archivos de forma sencilla.\\
Multiplataforma:\\
JabRef: Una interfaz libre y portable, escrita en Java para administrar referencias en formato BibTeX.\\
MAC\\
BibDesk: Un aplicación para Mac OS X para administrar referencias.\\

\end{frame}

\section{Estilos en BibTeX}
\begin{frame}{Estilos en BibTeX}
BiBTeX se acompaña de cuatro estilos llamados:
\begin{itemize}
\item plain
\item abbrv
\item alpha
\item unsrt
\end{itemize} 

\end{frame}

\begin{frame}{Estilos en BibTeX}
plain\\
    La lista bibliográfica final se ordena alfabéticamente atendiendo a la autoría, y si hubiera más de una obra del mismo autor, se toma en cuenta al año de las mismas y después el título. Si sigue habiendo igualdad tras aplicar los criterios anteriores, el último criterio es el del orden en el que fueron citadas y, para obras citadas simultáneamente mediante “\nocite{*}”, el orden que tengan en la base de datos.\\
    Las obras incluidas en la lista son numeradas consecutivamente y el número asignado a cada una de ellas, entre corchetes, se convierte en la etiqueta identificativa de la misma que será impresa en el lugar en el que se encuentren los comandos cite existentes en el cuerpo del documento.\\
    Los datos de los campos se incluyen completos.\\
    Para ciertos campos se añaden determinadas palabras o abreviaturas en inglés. Por ejemplo, tras el contenido del campo «edition» se añade la palabra “edition”, y hay tipos de registro en los que el nombre del campo (en inglés) forma parte de la referencia.\\
    
\end{frame}

\begin{frame}{Estilos en BibTeX}
bbrv \\
Es idéntico al estilo «plain» salvo en el hecho de que para ciertos datos se usan abreviaturas y así el nombre de pila de los autores es sustituido por sus iniciales y el nombre de ciertas revistas (que están predefinidas en el estilo y que se refieren a la informática) es sustituido por su abreviatura.\\
Para la mayor parte de los usuarios, que no usan las revistas predefinidas en el estilo, la única diferencia con «plain» es que del nombre propio del autor sólo se usan las iniciales.\\
\end{frame}

\begin{frame}{Estilos en BibTeX}
alpha \\
Se distingue del estilo «plain» exclusivamente en el hecho de que la etiqueta de identificación de cada obra en la lista no es un número, sino un texto generado automáticamente a partir de la autoría de la referencia, el año de publicación y, en ocasiones, el inicio del título.\\
Asimismo la lista bibliográfica se ordena alfabéticamente según las etiquetas asignadas, y el comando cite imprime, en el lugar en el que se encuentre, la etiqueta asignada a la obra citada.\\
\end{frame}

\begin{frame}{Estilos en BibTeX}
unsrt \\
Es igual al estilo «plain» salvo en el hecho de que en él la lista bibliográfica no se ordena alfabéticamente, sino según el orden en el que las distintas obras que aparecen en ella fueron citadas por primera vez.
En este caso, para las obras incluidas en la lista de referencias mediante un comando nocite se considerará que fueron citadas en el lugar en el que se encuentre tal comando. Y para el comando nocite{*}, se usará el orden en el que las referencias se encuentren en el fichero «.bib», pero sólo para las referencias que no hubieran sido citadas antes de nocite{*}.

\end{frame}

\section{Inserción en un documento}
\begin{frame}{Inserción en un documento}
Partimos que ya tenemos generado nuestra recopilación de citas.\\
En el lugar del documento principal en el que queramos que aparezca la lista con las referencias bibliográficas, debemos insertar las siguientes dos órdenes de LaTeX:\\
bibliography{MiBiblio}
bibliographystyle{MiEstilo}

\end{frame}

\section{Inserción en un documento}
\begin{frame}{Inserción en un documento}
Los comandos a utilizar para introducir una cita son los siguientes:\\
    El comando cite DatosAdicionales  clave  produce un doble efecto. En primer lugar, la referencia bibliográfica identificada en la base de datos mediante la clave recibida como parámetro, se incluirá en la lista bibliográfica. En segundo lugar, en el punto del documento donde se encontrara el comando, se imprimirá la etiqueta asignada a tal referencia en la lista de referencias junto con los datos adicionales que eventualmente hayamos incluido en el argumento opcional del comando.\\
    El comando nocite{clave} produce el primero de los efectos indicados, pero no el segundo, es decir: la obra identificada por la clave será incluida en la lista bibliográfica final, y en ella se le asignará asimismo una etiqueta (como a todas las obras de la lista), pero en el lugar del documento en el que se encuentra el comando nocite no se imprimirá nada.\\

\end{frame}

\section{Inserción en un documento}
\begin{frame}{Inserción en un documento}
Vamos a generar la lista de referencias bibliográficas:\\
Una vez que nuestro documento (llamémosle «HolaMundo.tex») está preparado, en los términos que se acaban de exponer, debemos seguir los siguientes pasos:\\
 

\end{frame}

\section{Inserción en un documento}
\begin{frame}{Inserción en un documento}
Compilar con LaTeX nuestro documento «.tex». Ello hará que se genere un fichero de extensión «.aux», en el que se incluirá información sobre la base de datos a usar, el fichero de estilo a usar, y las referencias bibliográficas que hay que incluir en la lista de referencias.\\
\end{frame}

\section{Inserción en un documento}
\begin{frame}{Inserción en un documento}
   
    Ejecutar, desde la línea de comandos, la orden «bibtex HolaMundo». Ello hará que BiBTeX lea el fichero «.aux», generado por la anterior compilación del documento, extrayendo de él la información que necesita para trabajar: qué base de datos debe usar, qué estilo, y qué referencias hay que buscar en la base.
    Y así, tras extraer de la base de datos los registros precisos, y formatearlos de acuerdo con el estilo indicado, BiBTeX genera un fichero de extensión «.bbl» en el que se contienen los comandos de LaTeX necesarios para escribir la lista de referencias bibliográficas que hay que insertar en el documento principal.
    BiBTeX genera también un fichero adicional, de extensión «.blg» que es un fichero «.log»

\end{frame}

\section{Inserción en un documento}
\begin{frame}{Inserción en un documento}
Compilar de nuevo el documento principal con LaTeX. En esta segunda compilación, al leer el comando bibliography, se insertará en su lugar el contenido del fichero «.bbl» generado en el paso anterior. Asimismo la nueva compilación reescribe el fichero «.aux», añadiendo a la información que ya existía en él la generada ahora, que es más completa pues incluye los datos de la lista bibliográfica final que se acaba de insertar en el documento.
\end{frame}

\section{Inserción en un documento}
\begin{frame}{Inserción en un documento}
Esta última información es usada en una nueva compilación con LaTeX para escribir correctamente los rótulos que hay que colocar en lugar de los comandos cite. Y eventualmente puede ser necesaria una nueva compilación: cuando alguno de los campos de la base de datos contenga algún comando de LaTeX que implique el uso de referencias cruzadas. En particular, el comando cite.


\end{frame}
\end{document}